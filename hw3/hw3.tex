\documentclass[12pt]{article}

\usepackage{setspace}
\usepackage{minted}
\usepackage{fancyvrb}
\usepackage{listings}

\usepackage[margin=0.75in]{geometry}
\pagestyle{empty}

\def \name       {Enrique Gavidia}
\def \coursenum  {CSC 415.01}
\def \coursename {Operating Systems Principles}
\def \instructor {Prof. Murphy}
\def \semester   {Spring 2012}
\def \assignment {Homework \# 3}
\def \duedate    {March 5, 2012}

\newcommand {\mytilde} {$\sim$}
\newcommand {\comment}[1] {\textcolor{red}{#1}}
\newcommand {\filename}[1] {\flushleft \textbf{#1}}
\newcommand {\append}[2] {\section*{Appendix #1} \textsl{\large #2}}

\newcommand {\makecover} {
  \begin{titlepage}
    \begin{center}
     
      \LARGE{\coursenum, \semester \\ \coursename}\\
      \Large{\instructor}\\
     
      \vfill
      \textbf{\Huge \assignment}\\
      \vfill
      
      \Large{\name}\\ 
      \large{\duedate}
      
    \end{center}
  \end{titlepage}
}

\DefineVerbatimEnvironment {shelloutput} {Verbatim} {fontsize=\scriptsize, numbers=left, frame=lines, commandchars=\%\{\}}

\newcommand {\includesource}[2] {\inputminted[linenos, fontsize=\scriptsize, frame=lines]{#1}{#2}}
\newcommand {\includeoutput}[1] {\VerbatimInput[fontsize=\scriptsize, numbers=left, frame=lines, commandchars=\%\{\}]{#1}}

\begin{document}

\makecover

%---{ Main Content }----------------------------------------
\section*{Assignment Description}
In this assignment, the goal was to create an interactive shell interface on top of the native Unix and Windows shell environments, that
demonstrated basic concurrency and process management. For the Unix environment, the shell takes advantage of the operating system's 
\texttt{PATH} variable to search for executable commands, making it a rather useful shell interface. On a Windows system however, the shell 
is only able to search for executables in the current directory, making it a bit limited for actual use. In both cases, the commands are 
executed (with all of their intended arguments) as child processes of the parent shell process, and the user has the option to set these commands 
to run in the background using the `\&` character in traditional Unix fashion. When set to run in the background, a process executes without 
interrupting the user's interaction with the shell; the child is allowed to finish its execution at its own pace, but launches the user straight 
back into the shell prompt to run more commands. 

This code has been tested to work under \textsl{Windows}, \textsl{Gentoo Linux}, and \textsl{Mac OSX}. 
The source for the Unix implementation under listed in \textbf{Appendix I}, and \textbf{Appendix III} for the Windows version.

\section*{Design \& Implementation}
I decided to keep the implementation of this shell as minimal and simple as possible. The entire program is essentially within the
confines of a simple 'read-eval-print-loop', where I perform basic operations to check whether the user input a special command (such as the
\texttt{exit}/\texttt{quit} commands) or entered a directive to spawn a background process. When it came to the operations performed inside 
this loop, I also kept it minimal, attempting to spawn child processes and execute commands only if provided with usable input that the shell itself 
can't directly handle (as in with \texttt{exit}). I also added a couple conveniences to the Windows implementation to make it easier to test. I added 
support for the native Windows \texttt{tasklist} command so that I could use it to demonstrate concurrency with background child processes within my 
same shell, much like how I could call the \texttt{ps} command in the Unix implementation. In addition to that, I also implemented support for optionally 
typing out executable names without their `.exe' extensions, like how is possible in the native Windows command prompt.
These minor conveniences I implemented on the Windows side were a great help to me when testing and debugging my code on that platform. In general however,
having quick printouts of the modified input string were my main means of debugging for both environments, as they clearly described what I was attempting
to create a child process for. 

\section*{Improvements}
This is a rather primitive shell implementation, so there are various ways in which I could improve it. As of now, there is no support for cursor movement
in the unix implementation, thus adding support for it would greatly improve it's usability. Tab-completion would also be a welcome addition for better 
usability. This shell is also lacking the ability to change directories, making it rather limited in usefulness. On the windows side, even though I have
automatic '.exe' insertion, it would still greatly benefit from gaining access to the system path to use actual windows commands instead of just searching
for executables in the current directory (which you cannot even change). And aside from piping and output redirection, a scripting interface like \texttt{Bash} or 
Windows' texttt{PowerShell} have, would also be a logical next step for improvement.

%---{ Appendices }------------------------------------------
\newpage

\append{I} {Unix Source Code}
\subsection*{Processes}
\filename{Version 1: Multiple \texttt{fprintf}'s}
\includesource{c}{unix_processes_v1.c}

\filename{Version 2: Multiple \texttt{fprintf}'s with \texttt{sleep}}
\includesource{c}{unix_processes_v2.c}

\filename{Version 3: Single \texttt{fprintf}}
\includesource{c}{unix_processes_v3.c}

\filename{Version 4: \texttt{fputs}}
\includesource{c}{unix_processes_v4.c}

\filename{Version 5: \texttt{sprintf} + \texttt{write} system-call}
\includesource{c}{unix_processes_v5.c}

\subsection*{Threads}


\append{II} {Unix Output}
\subsection*{Processes}
\filename{Version 1: Multiple \texttt{fprintf}'s}
\includeoutput{output/unix_processes_v1.txt}

\filename{Version 2: Multiple \texttt{fprintf}'s with \texttt{sleep}}
\includeoutput{output/unix_processes_v2.txt}
 
\filename{Version 3: Single \texttt{fprintf}}
\includeoutput{output/unix_processes_v3.txt}
 
\filename{Version 4: \texttt{fputs}}
\includeoutput{output/unix_processes_v4.txt}
 
\filename{Version 5: \texttt{sprintf} + \texttt{write} system-call}
\includeoutput{output/unix_processes_v5.txt}


\subsection*{Threads}


\append{III} {Windows Source Code}
\subsection*{Processes}

\filename{Parent Program}
\includesource{c}{win_processes_main.c}

\filename{Child Program}
\includesource{c}{win_processes.c}


\subsection*{Threads}


\append{IV} {Windows Output}
\subsection*{Processes}
\filename{Version 1: Multiple \texttt{fprintf}'s}
%\includeoutput{output/win_processes_v1.txt}

\filename{Version 2: Multiple \texttt{fprintf}'s with \texttt{sleep}}
%\includeoutput{output/win_processes_v2.txt}
 
\filename{Version 3: Single \texttt{fprintf}}
%\includeoutput{output/win_processes_v3.txt}
 
\filename{Version 4: \texttt{fputs}}
%\includeoutput{output/win_processes_v4.txt}
 
\filename{Version 5: \texttt{sprintf} + \texttt{write} system-call}
%\includeoutput{output/win_processes_v5.txt}


\subsection*{Threads}


\end{document}
