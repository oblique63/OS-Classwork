\documentclass[12pt]{article}

\usepackage{setspace}
\usepackage{minted}
\usepackage{fancyvrb}
\usepackage{listings}

\usepackage[margin=0.75in]{geometry}
\pagestyle{empty}

\def \name       {Enrique Gavidia}
\def \coursenum  {CSC 415.01}
\def \coursename {Operating Systems Principles}
\def \instructor {Prof. Murphy}
\def \semester   {Spring 2012}
\def \assignment {Homework \# 3}
\def \duedate    {March 5, 2012}

\newcommand {\mytilde} {$\sim$}
\newcommand {\comment}[1] {\textcolor{red}{#1}}
\newcommand {\filename}[1] {\flushleft \textbf{#1}}
\newcommand {\append}[2] {\section*{Appendix #1} \textsl{\large #2}}

\newcommand {\makecover} {
  \begin{titlepage}
    \begin{center}
      \LARGE{\coursenum, \semester \\ \coursename}\\
      \Large{\instructor}\\
      \vfill
      \textbf{\Huge \assignment}\\
      \vfill
      \Large{\name}\\
      \large{\duedate}
    \end{center}
  \end{titlepage}
}

\DefineVerbatimEnvironment {shelloutput} {Verbatim} {fontsize=\scriptsize, numbers=left, frame=lines, commandchars=\%\{\}}

\newcommand {\includesource}[2] {\inputminted[linenos, fontsize=\scriptsize, frame=lines]{#1}{#2}}
\newcommand {\includeoutput}[1] {\VerbatimInput[fontsize=\scriptsize, numbers=left, frame=lines, commandchars=\%\{\}]{#1}}

\begin{document}

\makecover

%---{ Main Content }--------------------------------------------------------------------
\section*{Assignment Description}
This assignment was centered around the behavior of multiple concurrent processes and threads, and the subtle pitfalls which can introduce
unwanted race conditions into a program. The goal was to intentionally write programs with as many race conditions as possible, and observe
their output. For the Process portion, this involved printing the digits 1 through 10 to \texttt{stderr} in each child process; whereas in
the Thread program, the threads were to attempt and increment a shared global variable by 10 each time they were spawned. The printing of 
output for the Processes was implemented 5 different ways, to experiment and analyze what might be the best way to present output from
concurrent processes, and combat race conditions.

This code has been tested to work under \textsl{Windows}, \textsl{Gentoo Linux}, and \textsl{Mac OSX}.
The source for the Unix implementation under listed in \textbf{Appendix I}, and \textbf{Appendix III} for the Windows version.

\section*{Design \& Implementation}
In general, I kept these programs as minimal as possible, but due to time constraints, I followed two distinct implementation styles for the 
Windows and Unix versions. For the Unix portion, I implemented each version of the Processes experiment in a separate program,
whereas in my Windows implementation, I combined all the versions into one program that selectively executes the appropriate version via the
arguments passed to it.  This design decision came about due to the Win32 API's lack of separate \texttt{fork} and \texttt{exec} functions,
which requires separate programs for the child processes. Thus to minimize the amount of programs needed (i.e. 5 different child process implementations
with corresponding parent programs), I simply implemented a single main parent program that accepts the implementation version number, and a single child 
program that gets spawned repeatedly, which also accepts the version number as an argument to tell it which implementation to execute.  As far as testing
goes, it helped greatly that I formatted my output to be terse and tab-separated, and included the number of the process (i.e. the order in which it was
spawned) instead of just relying on the PID to determine when it was spawned/executed.


\section*{Improvements}
The main thing I would improve would be the unix implementation of the Process part, so that it is more of a self-contained test suite like the Windows 
implementation, instead of a group of lose files. I would also probably add in the option to perform all the version tests at once (including the Threading
program) to quickly compare the differences between each approach. Different types of additional concurrency/race condition tests would also be a good addition, 
especially ones involving database access or file I/O.


\section*{Printing To \texttt{stderr} From Concurrent Processes}
For the first part of the assignment, when exploring methods for printing to \texttt{stderr} from concurrent processes, each version demonstrated how race conditions
can be subtly induced or minimized. In the first version (using multiple \texttt{fprintf}'s), the process ID's are assigned incrementally in the order the processes 
were spawned by the parent, but the output suggests they were not executed in that same order; they did however, manage to print out their values mostly in order with
a few race conditions. When adding the \texttt{sleep} call in the second version, the processes did execute more sequentially (closer to the order in which they were spawned),
but their outputs were heavily interleaved, as the \texttt{sleep} calls gave all the threads a chance to print out their values before continuing on to the next \texttt{fprintf}
iteration. For the version with the single \texttt{fprintf}, the outputs are no longer interleaved, and the processes are executed more sequentially than they were before. Similarly,
the output with the \texttt{fputs} and \texttt{write} system-call versions also had no interleaving output, and each improved upon the previous when it came to the order the
processes were executed in.  The best approach to output to \texttt{stderr} from multiple concurrent processes seems to be any method which dumps the output all at once, so that you 
don't rely on multiple print statements which can manifest as confusing, interleaved output.

%---{ Appendices }-----------------------------------------------------------------------
\newpage

%----{ UNIX }----------------------------------------------------------------------------
\append{I} {Unix Source Code}

\subsection*{Processes}
\filename{Version 1: Multiple \texttt{fprintf}'s}
\includesource{c}{unix_processes_v1.c}

\filename{Version 2: Multiple \texttt{fprintf}'s with \texttt{sleep}}
\includesource{c}{unix_processes_v2.c}

\filename{Version 3: Single \texttt{fprintf}}
\includesource{c}{unix_processes_v3.c}

\filename{Version 4: \texttt{fputs}}
\includesource{c}{unix_processes_v4.c}

\filename{Version 5: \texttt{sprintf} + \texttt{write} system-call}
\includesource{c}{unix_processes_v5.c}

\subsection*{Threads}
\includesource{c}{unix_threads.c}


\append{II} {Unix Output}

\subsection*{Processes}
\filename{Version 1: Multiple \texttt{fprintf}'s}
\includeoutput{output/unix_processes_v1.txt}

\filename{Version 2: Multiple \texttt{fprintf}'s with \texttt{sleep}}
\includeoutput{output/unix_processes_v2.txt}

\filename{Version 3: Single \texttt{fprintf}}
\includeoutput{output/unix_processes_v3.txt}

\filename{Version 4: \texttt{fputs}}
\includeoutput{output/unix_processes_v4.txt}

\filename{Version 5: \texttt{sprintf} + \texttt{write} system-call}
\includeoutput{output/unix_processes_v5.txt}

\subsection*{Threads}
\includeoutput{output/unix_threads.txt}


%----{ WINDOWS }-------------------------------------------------------------------------
\append{III} {Windows Source Code}

\subsection*{Processes}
\filename{Parent Program}
\includesource{c}{win_processes_main.c}

\filename{Child Program}
\includesource{c}{win_processes.c}

\subsection*{Threads}
\includesource{c}{win_threads.c}


\append{IV} {Windows Output}

\subsection*{Processes}
\filename{Version 1: Multiple \texttt{fprintf}'s}
\includeoutput{output/win_processes_v1.txt}

\filename{Version 2: Multiple \texttt{fprintf}'s with \texttt{Sleep}}
\includeoutput{output/win_processes_v2.txt}

\filename{Version 3: Single \texttt{fprintf}}
\includeoutput{output/win_processes_v3.txt}

\filename{Version 4: \texttt{fputs}}
\includeoutput{output/win_processes_v4.txt}

\filename{Version 5: \texttt{sprintf} + \texttt{WriteFile} system-call}
\includeoutput{output/win_processes_v5.txt}

\subsection*{Threads}
\includeoutput{output/win_threads.txt}

\end{document}
