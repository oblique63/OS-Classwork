\documentclass[12pt]{article}

\usepackage{setspace}
\usepackage{minted}
\usepackage{fancyvrb}
\usepackage{listings}

\usepackage[margin=0.75in]{geometry}
\pagestyle{empty}

\def \name       {Enrique Gavidia}
\def \coursenum  {CSC 415.01}
\def \coursename {Operating Systems Principles}
\def \instructor {Prof. Murphy}
\def \semester   {Spring 2012}
\def \assignment {Homework \# 4}
\def \duedate    {April 4, 2012}

\newcommand {\mytilde} {$\sim$}
\newcommand {\comment}[1] {\textcolor{red}{#1}}
\newcommand {\filename}[1] {\flushleft \textbf{#1}}
\newcommand {\append}[2] {\section*{Appendix #1} \textsl{\large #2}}

\newcommand {\makecover} {
  \begin{titlepage}
    \begin{center}
      \LARGE{\coursenum, \semester \\ \coursename}\\
      \Large{\instructor}\\
      \vfill
      \textbf{\Huge \assignment}\\
      \vfill
      \Large{\name}\\
      \large{\duedate}
    \end{center}
  \end{titlepage}
}

\DefineVerbatimEnvironment {shelloutput} {Verbatim} {fontsize=\scriptsize, numbers=left, frame=lines, commandchars=\%\{\}}

\newcommand {\includesource}[2] {\inputminted[linenos, fontsize=\scriptsize, frame=lines]{#1}{#2}}
\newcommand {\includeoutput}[1] {\VerbatimInput[fontsize=\scriptsize, numbers=left, frame=lines, commandchars=\%\{\}]{#1}}

\begin{document}

\makecover

%---{ Main Content }--------------------------------------------------------------------
\section*{Assignment Description}
For this assignment, we were tasked with implementing a solution to the 'Producer-Consumer' problem, to gain further insight into the
usage of semaphores for synchronizing concurrent tasks.  First we had to implement the basic data structure that is necessary to tackle
the problem, which is a queue of buffers. This data structure, is then  manipulated by multiple concurrent threads all trying to either add or remove
values to the queue in traditional 'Last-In-First-Out' fashion. Those threads trying to add to the queue were the 'Producers', and those removing
buffers from the queue were the 'Consumers' in this scenario.  The problem then lies in managing these threads so that they access and manipulate
the buffer queue in an orderly fashion, that has the Consumers consume after the Producers produce, and vice versa.  Thus we had to implement 'enqueue' 
and 'dequeue' methods that effectively prevent race conditions by means of counting semaphores and a mutex. The outline for this synchronization
strategy is documented in the \textsl{Pseudo Code} in \textbf{Appendix I}.

This code has been tested to work under \textsl{Windows}, \textsl{Gentoo Linux}, and \textsl{Mac OSX}.
The source for the Unix implementation under listed in \textbf{Appendix II}, and \textbf{Appendix V} for the Windows version.


\section*{Design \& Implementation}
I attempted to design this implementation with an emphasis on modularity, having my \texttt{BufferQueue} structure be shared across the synchronized
and unsyncronized versions of the program. I also kept a very clear separation of tasks between the functions, letting my \texttt{enqueue} and \texttt{dequeue} 
functions handle all the synchronization needed to manipulate the \texttt{BufferQueue} smoothly, while letting my \texttt{producer} and \texttt{consumer}
methods handle the threading and exit conditions. In order to achieve this, I had to store a local copy of the buffer queue's counter variable in each of the
threads, to ensure that race conditions didn't have an effect on the exit conditions needed to terminate the threads.


\section*{Testing Strategy}
The testing strategy I used in developing this implementation, was to simply disperse \texttt{fprintf} calls to \texttt{stderr} whenever a \texttt{enqueue} or
\texttt{dequeue} operation was performed, outputting the current number of filled buffers, and the contents of each buffer that was \texttt{enqueue}'d and 
\texttt{dequeue}'d. With this information, I could visually scan and determine the order in which the operations were executing, and whether or not the threads 
were performing their tasks properly (i.e. checking if the Consumers were consuming the values produced by the Producers in 'LIFO' order).


\section*{Improvements}
The main way I would improve this implementation, would be to unify all the \texttt{BufferQueue} related functions in the \texttt{BufferQueue} header file itself,
giving the functions options to either use synchronization or not, and to augment the header file with C-Pre-Processor directives to determine whether to use the 
Windows implementation, or the Unix implementation of the \texttt{BufferQueue}.  I would also probably implement proper queue and buffer structures for the 
\texttt{BufferQueue}, as opposed to the current implementation of a plain integer array with fancy \texttt{enqueue} and \texttt{dequeue} methods.


%---{ Appendices }-----------------------------------------------------------------------
\newpage
%----{ Pseudo Code }---------------------------------------------------------------------
\append{I} {Pseudo Code}

\filename{Buffer Queue}
\begin{scriptsize}
\begin{verbatim}

BufferQueue {
    Buffer queue[];
    int size;
    int count;

    int in;
    int out;

    // Counting semaphores
    Semaphore empty;
    Semaphore full;

    // Mutex lock
    Mutex lock;
}

enqueue(buffers) {
    // Wait for some empty buffers
    P(BufferQueue.empty);
    // Lock the mutex
    P(BufferQueue.lock);

    for (buffer in buffers) {
        BufferQueue.queue[BufferQueue.in] = buffer;
        BufferQueue.in = (BufferQueue.in+1) % BufferQueue.size;
        BufferQueue.count++;
    }

    // Unlock the mutex
    V(BufferQueue.lock);
    // Signal the new filled buffers
    V(BufferQueue.full);
}

dequeue(buffers) {
    // Wait for some buffers with values
    P(BufferQueue.full);
    // Lock the mutex
    P(BufferQueue.lock);

    for (buffer in buffers) {
        BufferQueue.queue[BufferQueue.out] = NULL;
        BufferQueue.out = (BufferQueue.out+1) % BufferQueue.size;
        BufferQueue.count--;
    }
    
    // Unlock the mutex
    V(BufferQueue.lock);
    // Signal the new empty buffers
    V(BufferQueue.empty);
}

\end{verbatim}
\end{scriptsize}

\filename{Producer + Consumer}
\begin{scriptsize}
\begin{verbatim}
producer() {
    do {
        spin_wait();

        enqueue(buffers_to_produce);
    } while (BufferQueue.count < BufferQueue.size);
}

consumer() {
    do {
        spin_wait();

        dequeue(buffers_to_consume);
    } while (BufferQueue.count > 0);
}
\end{verbatim}
\end{scriptsize}


%----{ UNIX }----------------------------------------------------------------------------
\append{II} {Unix Source Code}

\filename{Buffer Queue}
\includesource{c}{unix_buffer_queue.h}

\filename{Unsynchronized}
\includesource{c}{unix_unsynchronized.c}

\filename{Synchronized}
\includesource{c}{unix_synchronized.c}

\newpage
\append{III} {Unix Output}

\filename{Unsynchronized}
\includeoutput{output/unix_unsynchronized.txt}

\filename{Synchronized}
\includeoutput{output/unix_synchronized.txt}


%----{ STRACE }--------------------------------------------------------------------------
\append{IV} {Strace Output}

\filename{Unsynchronized}
\includeoutput{output/unix_unsynchronized_strace.txt}

\filename{Synchronized}
\includeoutput{output/unix_synchronized_strace.txt}

%----{ WINDOWS }-------------------------------------------------------------------------
\append{V} {Windows Source Code}

\filename{Buffer Queue}
\includesource{c}{win_buffer_queue.h}

\filename{Unsynchronized}
\includesource{c}{win_unsynchronized.c}

\filename{Synchronized}
\includesource{c}{win_synchronized.c}


\append{VI} {Windows Output}

\filename{Unsynchronized}
\includeoutput{output/win_unsynchronized.txt}

\filename{Synchronized}
\includeoutput{output/win_synchronized.txt}


\end{document}
