\documentclass[12pt]{article}

\usepackage{setspace}
\usepackage{minted}
\usepackage{fancyvrb}
\usepackage{listings}

\usepackage[margin=0.75in]{geometry}
\pagestyle{empty}

\def \name       {Enrique Gavidia}
\def \coursenum  {CSC 415.01}
\def \coursename {Operating Systems Principles}
\def \instructor {Prof. Murphy}
\def \semester   {Spring 2012}
\def \assignment {Homework \# 3A}
\def \duedate    {March 26, 2012}

\newcommand {\mytilde} {$\sim$}
\newcommand {\comment}[1] {\textcolor{red}{#1}}
\newcommand {\filename}[1] {\flushleft \textbf{#1}}
\newcommand {\append}[2] {\section*{Appendix #1} \textsl{\large #2}}

\newcommand {\makecover} {
  \begin{titlepage}
    \begin{center}
      \LARGE{\coursenum, \semester \\ \coursename}\\
      \Large{\instructor}\\
      \vfill
      \textbf{\Huge \assignment}\\
      \vfill
      \Large{\name}\\
      \large{\duedate}
    \end{center}
  \end{titlepage}
}

\DefineVerbatimEnvironment {shelloutput} {Verbatim} {fontsize=\scriptsize, numbers=left, frame=lines, commandchars=\%\{\}}

\newcommand {\includesource}[2] {\inputminted[linenos, fontsize=\scriptsize, frame=lines]{#1}{#2}}
\newcommand {\includeoutput}[1] {\VerbatimInput[fontsize=\scriptsize, numbers=left, frame=lines, commandchars=\%\{\}]{#1}}

\begin{document}

\makecover

%---{ Main Content }--------------------------------------------------------------------
\section*{Assignment Description}
This assignment built off our last one dealing with multiple concurrent processes and threads; however, now we had to prevent the race
conditions we induced in the last assignment. We were to keep the various design choices that led to the unsynchronized, interleaved output 
(such as the various \texttt{sleep} calls), and wrap them with signal and mutex based synchronization methods; as the goal was now to learn
about how to ensure that a concurrent program executes how we want it to.  The tasks performed by the concurrent child processes/threads 
(printing out the values from 1-10 [processes], and adding to 100 [threads]) now make up the critical sections of the code, and their execution
is now managed by semaphores (as described in the \textsl{Pseudo Code} in \textbf{Appendix I}), ensuring a synchronized sequential execution.
Additionally, we were also asked to induce deadlocks in our newly synchronized solutions, to point out another potential pitfall of concurrent
programming.

This code has been tested to work under \textsl{Windows}, \textsl{Gentoo Linux}, and \textsl{Mac OSX}.
The source for the Unix implementation under listed in \textbf{Appendix II}, and \textbf{Appendix IV} for the Windows version.

\section*{Design \& Implementation}
My goal with the design of my implementations was simply to be clear and concise about the mechanics involved, and what is going on in the code.
The basic logic for the synchronization closely follows the specifications laid out in the pseudo code for both the Unix and Windows implementations.
The only minor deviation is in the Windows deadlock version of the threaded program, where I actively lock the mutex in the parent before executing a
single child (and thus quickly invoke a \texttt{p()}- \texttt{p()} deadlock), as opposed to simply omitting the \texttt{v()} call to send a signal to the following process.
Aside from that, the Windows and Unix implementations of synchronous multi-threading are virtually identical.

% In general, I kept these programs as minimal as possible, but due to time constraints, I followed two distinct implementation styles for the 
% Windows and Unix versions. For the Unix portion, I implemented each version of the Processes experiment in a separate program,
% whereas in my Windows implementation, I combined all the versions into one program that selectively executes the appropriate version via the
% arguments passed to it.  This design decision came about due to the Win32 API's lack of separate \texttt{fork} and \texttt{exec} functions,
% which requires separate programs for the child processes. Thus to minimize the amount of programs needed (i.e. 5 different child process implementations
% with corresponding parent programs), I simply implemented a single main parent program that accepts the implementation version number, and a single child 
% program that gets spawned repeatedly, which also accepts the version number as an argument to tell it which implementation to execute.  As far as testing
% goes, it helped greatly that I formatted my output to be terse and tab-separated, and included the number of the process (i.e. the order in which it was
% spawned) instead of just relying on the PID to determine when it was spawned/executed.


\section*{Improvements}
The main improvement I would probably make, would be to rewrite the Unix process synchronization using a fuller semaphore construct, to make it 
clearer to see what is going on logically, as opposed to just using the raw \texttt{kill()} and \texttt{signal()} calls. If the programs were also
to simply stand on their own as examples of concurrent programming, I would take out the various unnecessary elements that were meant to invoke
race conditions in the previous assignment (such as the calls to \texttt{sleep()}), to prevent unnecessary confusion about the factors in play
when synchronizing multiple threads/processes.


% \section*{Printing To \texttt{stderr} From Concurrent Processes}
% For the first part of the assignment, when exploring methods for printing to \texttt{stderr} from concurrent processes, each version demonstrated how race conditions
% can be subtly induced or minimized. In the first version (using multiple \texttt{fprintf}'s), the process ID's are assigned incrementally in the order the processes 
% were spawned by the parent, but the output suggests they were not executed in that same order; they did however, manage to print out their values mostly in order with
% a few race conditions. When adding the \texttt{sleep} call in the second version, the processes did execute more sequentially (closer to the order in which they were spawned),
% but their outputs were heavily interleaved, as the \texttt{sleep} calls gave all the threads a chance to print out their values before continuing on to the next \texttt{fprintf}
% iteration. For the version with the single \texttt{fprintf}, the outputs are no longer interleaved, and the processes are executed more sequentially than they were before. Similarly,
% the output with the \texttt{fputs} and \texttt{write} system-call versions also had no interleaving output, and each improved upon the previous when it came to the order the
% processes were executed in.  The best approach to output to \texttt{stderr} from multiple concurrent processes seems to be any method which dumps the output all at once, so that you 
% don't rely on multiple print statements which can manifest as confusing, interleaved output.

%---{ Appendices }-----------------------------------------------------------------------
\newpage
%----{ Pseudo Code }---------------------------------------------------------------------
\append{I} {Pseudo Code}

\subsection*{Processes}

\begin{scriptsize}
\begin{verbatim}
semaphore s;
int pids[10];

main() {
    s = 1;
    for (index from 9 to 0) {
        pids[index] = fork();
        if (in child)
          break;
    }

    if (in child) {
        if (index == 0) {
           // do critical section task
           v(s);
        }
        else {
            p(s);
            // do critical section task
            v(s);
        }
    }
}
\end{verbatim}
\end{scriptsize}

\filename{Deadlocked Version}
\begin{scriptsize}
\begin{verbatim}
semaphore s;
int pids[10];

main() {
    s = 1;
    for (index from 9 to 0) {
        pids[index] = fork();
        if (in child)
          break;
    }

    if (in child) {
        if (index == 0) {
           // do critical section task
           v(s);
        }
        else {
            p(s);
            // do critical section task
            
            // don't signal rest of the child processes to start;
            // induce a p() -> p() deadlock
        }
    }
}
\end{verbatim}
\end{scriptsize}


\subsection*{Threads}

\begin{scriptsize}
\begin{verbatim}
semaphore s;
main() {
    s = 1;
    for (1 to 10)
        create_thread(do_task);
}

do_task() {
    p(s);
    // perform critical section task here
    v(s);
}
\end{verbatim}
\end{scriptsize}

\filename{Deadlocked Version}
\begin{scriptsize}
\begin{verbatim}
semaphore s;
main() {
    s = 1;
    for (1 to 10)
        create_thread(do_task);
}

do_task() {
    p(s);
    // perform critical section task here

    // don't reset the semaphore -- other threads forced to wait
    // for it to unlock, thus leading to a p()->p() deadlock
}
\end{verbatim}
\end{scriptsize}


%----{ UNIX }----------------------------------------------------------------------------
\append{II} {Unix Source Code}

\subsection*{Processes}
\includesource{c}{unix_processes.c}

\filename{Deadlocked Version}
\includesource{c}{unix_processes_deadlock.c}

\subsection*{Threads}
\includesource{c}{unix_threads.c}

\filename{Deadlocked Version}
\includesource{c}{unix_threads_deadlock.c}


\append{III} {Unix Output}

\subsection*{Processes}
\includeoutput{output/unix_processes.txt}

\filename{Deadlocked Version}
\includeoutput{output/unix_processes_deadlock.txt}

\subsection*{Threads}
\includeoutput{output/unix_threads.txt}

\filename{Deadlocked Version}
\includeoutput{output/unix_threads_deadlock.txt}


%----{ WINDOWS }-------------------------------------------------------------------------
\newpage
\append{IV} {Windows Source Code}

\subsection*{Threads}
\includesource{c}{win_threads.c}

\filename{Deadlocked Version}
\includesource{c}{win_threads_deadlock.c}


\append{V} {Windows Output}

\subsection*{Threads}
\includeoutput{output/win_threads.txt}

\filename{Deadlocked Version}
\includeoutput{output/win_threads_deadlock.txt}

\end{document}
