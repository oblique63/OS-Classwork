\documentclass[12pt]{article}

\usepackage{setspace}
\usepackage{minted}
\usepackage{fancyvrb}
\usepackage{listings}

\usepackage[margin=0.75in]{geometry}
\pagestyle{empty}

\def \name       {Enrique Gavidia}
\def \coursenum  {CSC 415.01}
\def \coursename {Operating Systems Principles}
\def \instructor {Prof. Murphy}
\def \semester   {Spring 2012}
\def \assignment {Homework \# 1}
\def \duedate    {February 8, 2012}

\newcommand {\mytilde} {$\sim$}
\newcommand {\comment}[1] {\textcolor{red}{#1}}
\newcommand {\filename}[1] {\flushleft \textbf{#1}}
\newcommand {\append}[2] {\section*{Appendix #1} \textsl{\large #2}}
\newcommand {\includesource}[2] {\inputminted[linenos, fontsize=\scriptsize, frame=lines]{#1}{#2}}

\DefineVerbatimEnvironment {shelloutput} {Verbatim} {fontsize=\scriptsize, numbers=left, commandchars=\#\{\}}

\begin{document}

%---{ Cover Page }------------------------------------------
\begin{titlepage}
  \begin{center}

    {\LARGE \coursenum, \semester \\ \coursename}\\
    {\Large \instructor}\\

    \vfill
    \textbf{\Huge \assignment}\\
    \vfill
    
    {\Large \name}\\ 
    {\large \duedate}
    
  \end{center}
\end{titlepage}


%---{ Main Content }----------------------------------------
\section*{Assignment Description}
This assignment was an introductory exercise in programming in C with low-level system calls in two contrasting operating 
system environments; a Unix derived system, and the Windows environment. The goal of the program was to make copies of files 
using the native operating system APIs, while providing a command line interface for users using C's formatted I/O library of
functions.  The interaction between the program and the user consists of a simple prompt asking the user which file is to be copied,
and what file the copy should be output to, followed by a printout of the file contents and the amount of bytes successfully copied.
The option to display debugging information is also provided, to demonstrate how the program is performing its read/write operations
via incremental buffer reads.  Similarly, the code is also reassured with extensive error handling features which display useful feedback
in the event that something goes wrong, and return the appropriate error codes to the operating system.

This code has been tested to work under \textsl{Windows}, \textsl{Gentoo Linux}, and \textsl{Mac OSX}. 
The source for the Unix implementation is listed in \textbf{Appendix I}, and \textbf{Appendix III} for the Windows version.


\section*{Design \& Implementation}
I wrote the entirety of this program with my own original code and decided to go with a rather modular approach in its design, delegating the key 
parts of the program to external methods outside the `main' function.  Although not entirely necessary given the simple nature of the requirements, 
it made it easier to experiment with certain approaches in isolation; i.e. once I established the user interface code, the organization made it easier 
to switch focus completely to the I/O portion of the assignment without worrying about `the whole'.  On a similar note, I also avoid using too many 
compound operations (such as performing an I/O call in the invariant of a loop) for the sake of readability.  When it comes to platform-specific 
eccentricities, such as the Win32 datatypes or Linux's strict `printf' implementation that requires a null-terminator, I made sure to document the code 
as clearly as possible to convey what is happening; either through comments, or descriptive variable names. 

Due to this emphasis on clear coding practice, debugging and testing were not too big of an issue, however for when they were necessary, I implemented
a debug flag option to be passed to the executable at runtime to easily switch between debugging and non-debugging modes. With the flag set, I would
monitor any variables and values that interested me via print statements, like the buffer dump debug option required by the assignment. 


\section*{Improvements}
To improve the program, I would essentially just make it more robust by double-checking to make sure that the file copied correctly instead of
having to rely of an external diff tool; by giving more detailed error messages; and by giving users the option to view and overwrite any existing 
copies of the output file, as opposed to just aborting the operation if the file exists.  Implementation-wise, I might also remove some of the extra 
code-bloat that came as a result of keeping the implementation overly modular and `DRY' (i.e. `get\_input' function that consists of only 2-lines). 


%---{ Appendices }------------------------------------------
\newpage

\append{I} {Unix Source Code}
\includesource{c}{unix_copy.c}


\append{II} {Unix Output}
\subsection*{Small Files}

\filename{smallfile1}
\begin{shelloutput}
=comment{The debug flag `-d' is enabled for the small files, to print out the debug statements}
oblique@netlab4 #mytilde/hw1 $ ./unix_copy -d
Welcome to the TEE Copy Program by Enrique Gavidia!
Enter the name of the file to copy from:
smallfile1
Enter the name of the file to copy to:
smallcopy1

#comment{The buffer dumps in the debug output contain a maximum of 11 characters, as was designated in `buffer\_size' }
--------
BUFFER: abcdefghijk
--------
abcdefghijk
--------
BUFFER: lmnopqrstuv
--------
#comment{The normal printout of the buffer has no additional formatting, so as to act as a continuous stream to `stdout'} 
#comment{and replicate the contents of the file exactly on the user's screen. Here however, it is being interrupted by }
#comment{the debug output.}
lmnopqrstuv
--------
BUFFER: wxyz
123456
--------
wxyz
123456
--------
BUFFER: 7890

--------
7890
File copy successful, 38 bytes copied

#comment{The `-s' flag makes diff report when files are the same}
oblique@netlab4 @-mytilde/hw1 $ diff -s smallfile1 smallcopy1
Files smallfile1 and smallcopy1 are identical
\end{shelloutput}


\filename{smallfile2}
\begin{shelloutput}
oblique@netlab4 #mytilde/hw1 $ ./unix_copy -d
Welcome to the TEE Copy Program by Enrique Gavidia!
Enter the name of the file to copy from:
smallfile2
Enter the name of the file to copy to:
smallcopy2

--------
BUFFER: This is a t
--------
This is a t
--------
BUFFER: est for the
--------
est for the
--------
BUFFER:  csc 415 fi
--------
 csc 415 fi
--------
BUFFER: le-copy ass
--------
le-copy ass
--------
BUFFER: ignment.

--------
ignment.
File copy successful, 53 bytes copied

oblique@netlab4 #mytilde/hw1 $ diff -s smallfile2 smallcopy2
Files smallfile2 and smallcopy2 are identical
\end{shelloutput}


\filename{smallfile3}
\begin{shelloutput}
oblique@netlab4 #mytilde/hw1 $ ./unix_copy -d
Welcome to the TEE Copy Program by Enrique Gavidia!
Enter the name of the file to copy from:
smallfile3
Enter the name of the file to copy to:
smallcopy3

--------
BUFFER: meow meow m
--------
meow meow m
--------
BUFFER: eow meow me
--------
eow meow me
--------
BUFFER: ow meow meo
--------
ow meow meo
--------
BUFFER: w meow

--------
w meow
File copy successful, 40 bytes copied

oblique@netlab4 #mytilde/hw1 $ diff -s smallfile3 smallcopy3
Files smallfile3 and smallcopy3 are identical
\end{shelloutput}



\subsection*{Large Files}

\filename{largefile1}
\begin{shelloutput}
oblique@netlab4 #mytilde/hw1 $ ./unix_copy
Welcome to the TEE Copy Program by Enrique Gavidia!
Enter the name of the file to copy from:
largefile1
Enter the name of the file to copy to:
largecopy1
#comment{larger files were executed without the debug flag, thus maintaining their original formatting when printed to stdout}
Delicious Black Bean Burritos
=============================

Ingredients

2 (10 inch) flour tortillas
2 tablespoons vegetable oil
1 small onion, chopped
1/2 red bell pepper, chopped
1 teaspoon minced garlic
1 (15 ounce) can black beans, rinsed and drained
1 teaspoon minced jalapeno peppers
3 ounces cream cheese
1/2 teaspoon salt
2 tablespoons chopped fresh cilantro


Directions

1) Wrap tortillas in foil and place in oven heated to 350 degrees F (175 degrees C). Bake for 15 minutes or until heated through.

2) Heat oil in a 10-inch skillet over medium heat. Place onion, bell pepper, garlic and jalapenos in skillet, cook for 2 minutes
stirring occasionally. Pour beans into skillet, cook 3 minutes stirring.

3) Cut cream cheese into cubes and add to skillet with salt. Cook for 2 minutes stirring occasionally. Stir cilantro into mixture.

4) Spoon mixture evenly down center of warmed tortilla and roll tortillas up. Serve immediately.
File copy successful, 946 bytes copied
oblique@netlab4 #mytilde/hw1 $ diff -s largefile1 largecopy1
Files largefile1 and largecopy1 are identical
\end{shelloutput}


\filename{largefile2}
\begin{shelloutput}
oblique@netlab4 #mytilde/hw1 $ ./unix_copy
Welcome to the TEE Copy Program by Enrique Gavidia!
Enter the name of the file to copy from:
largefile2
Enter the name of the file to copy to:
largecopy2
Metallica - Master Of Puppets
|--------------------------------------------------------------------|
|--------------------------------------------------------------------|
|--------------------------------------------------------------------|
|--------------------------------------------------------------------|
|-----2-----3-----4-----3-----2-2-----2-----3------5\4---5\4---5\4---|
|-0-1---0-1---0-1---0-1---0-1-----0-1---0-1---0-0--3\2-0-3\2-0-3\2---|

|--------------------------------------------------------------------|
|--------------------------------------------------------------------|
|-------------------------------------------------------------4------|
|-------------------------------------------------------------4------|
|-----2-----3-----4-----3-----2-2-----2-----3-----4-----3-----2------|
|-0-1---0-1---0-1---0-1---0-1-----0-1---0-1---0-1---0-1---0-1--------|

File copy successful, 884 bytes copied
oblique@netlab4 #mytilde/hw1 $ diff -s largefile2 largecopy2
Files largefile2 and largecopy2 are identical

\end{shelloutput}


\filename{largefile3}
\begin{shelloutput}
oblique@netlab4 #mytilde/hw1 $ ./unix_copy
Welcome to the TEE Copy Program by Enrique Gavidia!
Enter the name of the file to copy from:
largefile3
Enter the name of the file to copy to:
largecopy3
!!!!!!!!!!!!!!!!!!!!!!!!!!!!!!!!!!!!!!!!!!!!!!!!!!!!!!!!!!!!!!!!!!!!!!!!!!
!!!!!!!!!!!!!!!!!!!!!!!!!!!!!!!!!!!!!!!`   `4!!!!!!!!!!~4!!!!!!!!!!!!!!!!!
!!!!!!!!!!!!!!!!!!!!!!!!!!!!!!!!!!!!!   <~:   ~!!!~   ..  4!!!!!!!!!!!!!!!
!!!!!!!!!!!!!!!!!!!!!!!!!!!!!!!!!!!!  ~~~~~~~  '  ud$$$$$  !!!!!!!!!!!!!!!
!!!!!!!!!!!!!!!!!!!!!!!!!!!!!!!!!!!  ~~~~~~~~~: ?$$$$$$$$$  !!!!!!!!!!!!!!
!!!!!!!!!!!`     ``~!!!!!!!!!!!!!!  ~~~~~          "*$$$$$k `!!!!!!!!!!!!!
!!!!!!!!!!  $$$$$bu.  '~!~`     .  '~~~~      :~~~~          `4!!!!!!!!!!!
!!!!!!!!!  $$$$$$$$$$$c  .zW$$$$$E ~~~~      ~~~~~~~~  ~~~~~:  '!!!!!!!!!!
!!!!!!!!! d$$$$$$$$$$$$$$$$$$$$$$E ~~~~~    '~~~~~~~~    ~~~~~  !!!!!!!!!!
!!!!!!!!> 9$$$$$$$$$$$$$$$$$$$$$$$ '~~~~~~~ '~~~~~~~~     ~~~~  !!!!!!!!!!
!!!!!!!!> $$$$$$$$$$$$$$$$$$$$$$$$b   ~~~    '~~~~~~~     '~~~ '!!!!!!!!!!
!!!!!!!!> $$$$$$$$$$$$$$$$$$$$$$$$$$$cuuue$$N.   ~        ~~~  !!!!!!!!!!!
!!!!!!!!! **$$$$$$$$$$$$$$$$$$$$$$$$$$$$$$$$$$$$Ne  ~~~~~~~~  `!!!!!!!!!!!
!!!!!!!!!  J$$$$$$$$$$$$$$$$$$$$$$$$$$$$$$$$$$$$$$N  ~~~~~  zL '!!!!!!!!!!
!!!!!!!!  d$$$$$$$$$$$$$$$$$$$$$$$$$$$$$$$$$$$$$$$$$c     z$$$c `!!!!!!!!!
!!!!!!!> <$$$$$$$$$$$$$$$$$$$$$$$$$$$$$$$$$$$$$$$$$$$$$$$$$$$$$> 4!!!!!!!!
!!!!!!!  $$$$$$$$$$$$$$$$$$$$$$$$$$$$$$$$$$$$$$$$$$$$$$$$$$$$$$$  !!!!!!!!
!!!!!!! <$$$$$$$$$$$$$$$$$$$$$$$$$$$$$$$$$$$$$$$$$$$$$$$$$$$$$*"   ....:!!
!!!!!!~ 9$$$$$$$$$$$$$$$$$$$$$$$$$$$$$$$$$$$$$$$$$$$$$$$$$$$$e@$N '!!!!!!!
!!!!!!  9$$$$$$$$$$$$$$$$$$$$$$$$$$$$$$$$$$$$$$$$$$$$$$$$$$$$$$$$  !!!!!!!
!!!!!!  $$$$$$$$$$$$$$$$$$$$$$$$$$$$$$$$$$$$$$$$$$$$""$$$$$$$$$$$~ ~~4!!!!
!!!!!!  9$$$$$$$$$$$$$$$$$$$$$$$$$$$$$$$$$$$$$$$$$$    $$$$$$$Lue  :::!!!!
!!!!!!> 9$$$$$$$$$$$$" '$$$$$$$$$$$$$$$$$$$$$$$$$$$    $$$$$$$$$$  !!!!!!!
!!!!!!! '$$*$$$$$$$$E   '$$$$$$$$$$$$$$$$$$$$$$$$$$$u.@$$$$$$$$$E '!!!!!!!
!!!!~`   .eeW$$$$$$$$   :$$$$$$$$$$$$$***$$$$$$$$$$$$$$$$$$$$u.    `~!!!!!
!!> .:!h '$$$$$$$$$$$$ed$$$$$$$$$$$$Fz$$b $$$$$$$$$$$$$$$$$$$$$F '!h.  !!!
!!!!!!!!L '$**$$$$$$$$$$$$$$$$$$$$$$ *$$$ $$$$$$$$$$$$$$$$$$$$F  !!!!!!!!!
!!!!!!!!!   d$$$$$$$$$$$$$$$$$$$$$$$$buud$$$$$$$$$$$$$$$$$$$$"  !!!!!!!!!!
!!!!!!! .<!  #$$*"$$$$$$$$$$$$$$$$$$$$$$$$$$$$$$$$$$$$$$$$$*  :!!!!!!!!!!!
!!!!!!!!!!!!:   d$$$$$$$$$$$$$$$$$$$$$$$$$$$$$$$$$$$$$$$$#  :!!!!!!!!!!!!!
!!!!!!!!!!!~  :  '#$$$$$$$$$$$$$$$$$$$$$$$$$$$$$$$$$$*"    !!!!!!!!!!!!!!!
!!!!!!!!!!  !!!!!:   ^"**$$$$$$$$$$$$$$$$$$$$**#"     .:<!!!!!!!!!!!!!!!!!
!!!!!!!!!!!!!!!!!!!!!:...                      .::!!!!!!!!!!!!!!!!!!!!!!!!
!!!!!!!!!!!!!!!!!!!!!!!!!!!!!!!!!!!!!!!!!!!!!!!!!!!!!!!!!!!!!!!!!!!!!!!!!!
File copy successful, 2550 bytes copied
oblique@netlab4 #mytilde/hw1 $ diff -s largefile3 largecopy3
Files largefile3 and largecopy3 are identical
\end{shelloutput}



\subsection*{Strace}
\begin{shelloutput}
#comment{The `-e' flag enables the ability to tell which system calls strace should watch}
oblique@netlab4 #mytilde/hw1 $ strace -e trace=open,read,write  ./unix_copy
open("/etc/ld.so.cache", O_RDONLY)      = 3
open("/lib/libc.so.6", O_RDONLY)        = 3
read(3, "\\177ELF\\1\\1\\1\\0\\\0\\\0\0\0\0\0\0\0\3\0\3\0\1\0\0\0\200\242\1\0004\0\0\0"..., 512
) = 512
#comment{This is the first system call in the file; a write to stdout via `puts'}
write(1, "Welcome to the TEE Copy Program "..., 52Welcome to the TEE Copy Program by Enrique
 Gavidia!
) = 52
write(1, "Enter the name of the file to co"..., 41Enter the name of the file to copy from:
) = 41

#comment{Here is the read call to stdin performed by `sprintf'}
read(0, Makefile
"Makefile\n", 1024)             = 9
write(1, "Enter the name of the file to co"..., 39Enter the name of the file to copy to:
) = 39
read(0, Makefile_copy
"Makefile_copy\n", 1024)        = 14

#comment{These are the open system calls that were performed directly by the code once the file names were obtained}
open("Makefile", O_RDONLY)              = 3
open("Makefile_copy", O_WRONLY|O_CREAT, 0600) = 4

#comment{Here is where the read+write loop starts, and as the strace output demonstrates, the calls were each reading}
#comment{and writing 11 bytes, the size designated by the `buffer_size' constant}
read(3, "CFLAGS=-Wal", 11)              = 11
write(4, "CFLAGS=-Wal", 11)             = 11
read(3, "l -Wextra -", 11)              = 11
write(4, "l -Wextra -", 11)             = 11
read(3, "g\n\nFILES = ", 11)            = 11
write(4, "g\n\nFILES = ", 11)           = 11
write(1, "CFLAGS=-Wall -Wextra -g\n\n", 25CFLAGS=-Wall -Wextra -g

) = 25
read(3, "unix_copy\n\n", 11)            = 11
write(4, "unix_copy\n\n", 11)           = 11
write(1, "FILES = unix_copy\n\n", 19FILES = unix_copy

)   = 19
read(3, "all: $(FILE", 11)              = 11
write(4, "all: $(FILE", 11)             = 11
read(3, "S)\n\nclean:\n", 11)           = 11
write(4, "S)\n\nclean:\n", 11)          = 11
write(1, "all: $(FILES)\n\nclean:\n", 22all: $(FILES)

clean:
) = 22
read(3, "\trm -f $(FI", 11)             = 11
write(4, "\trm -f $(FI", 11)            = 11
read(3, "LES)\n\nwith-", 11)            = 11
write(4, "LES)\n\nwith-", 11)           = 11
write(1, "\trm -f $(FILES)\n\n", 17	rm -f $(FILES)

)    = 17
read(3, "clean: clea", 11)              = 11
write(4, "clean: clea", 11)             = 11
read(3, "n\n\tmake\n", 11)              = 8
write(4, "n\n\tmake\n", 8)              = 8
write(1, "with-clean: clean\n\tmake\n", 24with-clean: clean
	make
) = 24
read(3, "", 11)                         = 0
#comment{The loop then terminates upon reading/writing 0 bytes, and the code proceeds to print out the total number of}
#comment{bytes read via `printf'}
write(4, "", 0)                         = 0
write(1, "File copy successful, 107 bytes "..., 39File copy successful, 107 bytes copied
) = 39
\end{shelloutput}


\append{III} {Windows Source Code}
\includesource{c}{win_copy.c}


\append{IV} {Windows Output}
\subsection*{Small Files}

\filename{smallfile1}
\begin{shelloutput}
C:\Users\Enrique\School\OS-Classwork\hw1>win_copy -d
Welcome to the TEE Copy Program by Enrique Gavidia!
Enter the name of the file to copy from:
smallfile1
Enter the name of the file to copy to:
smallcopy1

--------
BUFFER: abcdefghijk
abcdefghijk
--------
BUFFER: lmnopqrstuv
lmnopqrstuv
--------
BUFFER: wxyz
123456
wxyz
123456
--------
BUFFER: 7890
7890
--------
BUFFER:

File copy successful, 37 bytes copied

#comment{The windows output is the same as the Unix output, as it should be, however to verify that}
#comment{the files are in fact identical, the windows equivalent of the `diff' command is used: `comp'.}
#comment{And similarly to `diff', it doesn't output very much when the files are indeed identical.}
C:\Users\Enrique\School\OS-Classwork\hw1>comp smallfile1 smallcopy1
Comparing smallfile1 and smallcopy1...
Files compare OK

\end{shelloutput}



\filename{smallfile2}
\begin{shelloutput}
C:\Users\Enrique\School\OS-Classwork\hw1>win_copy -d
Welcome to the TEE Copy Program by Enrique Gavidia!
Enter the name of the file to copy from:
smallfile2
Enter the name of the file to copy to:
smallcopy2

--------
BUFFER: this is a t
this is a t
--------
BUFFER: est for the
est for the
--------
BUFFER:  csc 415 fi
 csc 415 fi
--------
BUFFER: le-copy ass
le-copy ass
--------
BUFFER: ignment.
ignment.
--------
BUFFER:

File copy successful, 52 bytes copied

C:\Users\Enrique\School\OS-Classwork\hw1>comp smallfile2 smallcopy2
Comparing smallfile2 and smallcopy2...
Files compare OK
\end{shelloutput}


\filename{smallfile3}
\begin{shelloutput}
C:\Users\Enrique\School\OS-Classwork\hw1>win_copy -d
Welcome to the TEE Copy Program by Enrique Gavidia!
Enter the name of the file to copy from:
smallfile3
Enter the name of the file to copy to:
smallcopy3

--------
BUFFER: meow meow m
meow meow m
--------
BUFFER: eow meow me
eow meow me
--------
BUFFER: ow meow meo
ow meow meo
--------
BUFFER: w meow
w meow
--------
BUFFER:

File copy successful, 39 bytes copied

C:\Users\Enrique\School\OS-Classwork\hw1>comp smallfile3 smallcopy3
Comparing smallfile3 and smallcopy3...
Files compare OK
\end{shelloutput}


\subsection*{Large Files}

\filename{largefile1}
\begin{shelloutput}
C:\Users\Enrique\School\OS-Classwork\hw1>win_copy
Welcome to the TEE Copy Program by Enrique Gavidia!
Enter the name of the file to copy from:
largefile1
Enter the name of the file to copy to:
largecopy1
Delicious Black Bean Burritos
=============================

Ingredients

2 (10 inch) flour tortillas
2 tablespoons vegetable oil
1 small onion, chopped
1/2 red bell pepper, chopped
1 teaspoon minced garlic
1 (15 ounce) can black beans, rinsed and drained
1 teaspoon minced jalapeno peppers
3 ounces cream cheese
1/2 teaspoon salt
2 tablespoons chopped fresh cilantro


Directions

1) Wrap tortillas in foil and place in oven heated to 350 degrees F (175 degrees C). Bake for 15 minutes or until heated thro
ugh.

2) Heat oil in a 10-inch skillet over medium heat. Place onion, bell pepper, garlic and jalapenos in skillet, cook for 2 minu
tes
stirring occasionally. Pour beans into skillet, cook 3 minutes stirring.

3) Cut cream cheese into cubes and add to skillet with salt. Cook for 2 minutes stirring occasionally. Stir cilantro into mix
ture.

4) Spoon mixture evenly down center of warmed tortilla and roll tortillas up. Serve immediately.
File copy successful, 971 bytes copied

C:\Users\Enrique\School\OS-Classwork\hw1>comp largefile1 largecopy1
Comparing largefile1 and largecopy1...
Files compare OK

\end{shelloutput}


\filename{largefile2}
\begin{shelloutput}
C:\Users\Enrique\School\OS-Classwork\hw1>win_copy
Welcome to the TEE Copy Program by Enrique Gavidia!
Enter the name of the file to copy from:
largefile2
Enter the name of the file to copy to:
largecopy2
Metallica - Master Of Puppets
|--------------------------------------------------------------------|
|--------------------------------------------------------------------|
|--------------------------------------------------------------------|
|--------------------------------------------------------------------|
|-----2-----3-----4-----3-----2-2-----2-----3------5\4---5\4---5\4---|
|-0-1---0-1---0-1---0-1---0-1-----0-1---0-1---0-0--3\2-0-3\2-0-3\2---|

|--------------------------------------------------------------------|
|--------------------------------------------------------------------|
|-------------------------------------------------------------4------|
|-------------------------------------------------------------4------|
|-----2-----3-----4-----3-----2-2-----2-----3-----4-----3-----2------|
|-0-1---0-1---0-1---0-1---0-1-----0-1---0-1---0-1---0-1---0-1--------|


File copy successful, 899 bytes copied

C:\Users\Enrique\School\OS-Classwork\hw1>comp largefile2 largecopy2
Comparing largefile2 and largecopy2...
Files compare OK

\end{shelloutput}


\filename{largefile3}
\begin{shelloutput}
C:\Users\Enrique\School\OS-Classwork\hw1>win_copy
Welcome to the TEE Copy Program by Enrique Gavidia!
Enter the name of the file to copy from:
largefile3
Enter the name of the file to copy to:
largecopy3
!!!!!!!!!!!!!!!!!!!!!!!!!!!!!!!!!!!!!!!!!!!!!!!!!!!!!!!!!!!!!!!!!!!!!!!!!!
!!!!!!!!!!!!!!!!!!!!!!!!!!!!!!!!!!!!!!!`   `4!!!!!!!!!!~4!!!!!!!!!!!!!!!!!
!!!!!!!!!!!!!!!!!!!!!!!!!!!!!!!!!!!!!   <~:   ~!!!~   ..  4!!!!!!!!!!!!!!!
!!!!!!!!!!!!!!!!!!!!!!!!!!!!!!!!!!!!  ~~~~~~~  '  ud$$$$$  !!!!!!!!!!!!!!!
!!!!!!!!!!!!!!!!!!!!!!!!!!!!!!!!!!!  ~~~~~~~~~: ?$$$$$$$$$  !!!!!!!!!!!!!!
!!!!!!!!!!!`     ``~!!!!!!!!!!!!!!  ~~~~~          "*$$$$$k `!!!!!!!!!!!!!
!!!!!!!!!!  $$$$$bu.  '~!~`     .  '~~~~      :~~~~          `4!!!!!!!!!!!
!!!!!!!!!  $$$$$$$$$$$c  .zW$$$$$E ~~~~      ~~~~~~~~  ~~~~~:  '!!!!!!!!!!
!!!!!!!!! d$$$$$$$$$$$$$$$$$$$$$$E ~~~~~    '~~~~~~~~    ~~~~~  !!!!!!!!!!
!!!!!!!!> 9$$$$$$$$$$$$$$$$$$$$$$$ '~~~~~~~ '~~~~~~~~     ~~~~  !!!!!!!!!!
!!!!!!!!> $$$$$$$$$$$$$$$$$$$$$$$$b   ~~~    '~~~~~~~     '~~~ '!!!!!!!!!!
!!!!!!!!> $$$$$$$$$$$$$$$$$$$$$$$$$$$cuuue$$N.   ~        ~~~  !!!!!!!!!!!
!!!!!!!!! **$$$$$$$$$$$$$$$$$$$$$$$$$$$$$$$$$$$$Ne  ~~~~~~~~  `!!!!!!!!!!!
!!!!!!!!!  J$$$$$$$$$$$$$$$$$$$$$$$$$$$$$$$$$$$$$$N  ~~~~~  zL '!!!!!!!!!!
!!!!!!!!  d$$$$$$$$$$$$$$$$$$$$$$$$$$$$$$$$$$$$$$$$$c     z$$$c `!!!!!!!!!
!!!!!!!> <$$$$$$$$$$$$$$$$$$$$$$$$$$$$$$$$$$$$$$$$$$$$$$$$$$$$$> 4!!!!!!!!
!!!!!!!  $$$$$$$$$$$$$$$$$$$$$$$$$$$$$$$$$$$$$$$$$$$$$$$$$$$$$$$  !!!!!!!!
!!!!!!! <$$$$$$$$$$$$$$$$$$$$$$$$$$$$$$$$$$$$$$$$$$$$$$$$$$$$$*"   ....:!!
!!!!!!~ 9$$$$$$$$$$$$$$$$$$$$$$$$$$$$$$$$$$$$$$$$$$$$$$$$$$$$e@$N '!!!!!!!
!!!!!!  9$$$$$$$$$$$$$$$$$$$$$$$$$$$$$$$$$$$$$$$$$$$$$$$$$$$$$$$$  !!!!!!!
!!!!!!  $$$$$$$$$$$$$$$$$$$$$$$$$$$$$$$$$$$$$$$$$$$$""$$$$$$$$$$$~ ~~4!!!!
!!!!!!  9$$$$$$$$$$$$$$$$$$$$$$$$$$$$$$$$$$$$$$$$$$    $$$$$$$Lue  :::!!!!
!!!!!!> 9$$$$$$$$$$$$" '$$$$$$$$$$$$$$$$$$$$$$$$$$$    $$$$$$$$$$  !!!!!!!
!!!!!!! '$$*$$$$$$$$E   '$$$$$$$$$$$$$$$$$$$$$$$$$$$u.@$$$$$$$$$E '!!!!!!!
!!!!~`   .eeW$$$$$$$$   :$$$$$$$$$$$$$***$$$$$$$$$$$$$$$$$$$$u.    `~!!!!!
!!> .:!h '$$$$$$$$$$$$ed$$$$$$$$$$$$Fz$$b $$$$$$$$$$$$$$$$$$$$$F '!h.  !!!
!!!!!!!!L '$**$$$$$$$$$$$$$$$$$$$$$$ *$$$ $$$$$$$$$$$$$$$$$$$$F  !!!!!!!!!
!!!!!!!!!   d$$$$$$$$$$$$$$$$$$$$$$$$buud$$$$$$$$$$$$$$$$$$$$"  !!!!!!!!!!
!!!!!!! .<!  #$$*"$$$$$$$$$$$$$$$$$$$$$$$$$$$$$$$$$$$$$$$$$*  :!!!!!!!!!!!
!!!!!!!!!!!!:   d$$$$$$$$$$$$$$$$$$$$$$$$$$$$$$$$$$$$$$$$#  :!!!!!!!!!!!!!
!!!!!!!!!!!~  :  '#$$$$$$$$$$$$$$$$$$$$$$$$$$$$$$$$$$*"    !!!!!!!!!!!!!!!
!!!!!!!!!!  !!!!!:   ^"**$$$$$$$$$$$$$$$$$$$$**#"     .:<!!!!!!!!!!!!!!!!!
!!!!!!!!!!!!!!!!!!!!!:...                      .::!!!!!!!!!!!!!!!!!!!!!!!!
!!!!!!!!!!!!!!!!!!!!!!!!!!!!!!!!!!!!!!!!!!!!!!!!!!!!!!!!!!!!!!!!!!!!!!!!!!
File copy successful, 2582 bytes copied

C:\Users\Enrique\School\OS-Classwork\hw1>comp largefile3 largecopy3
Comparing largefile3 and largecopy3...
Files compare OK

\end{shelloutput}

\end{document}
