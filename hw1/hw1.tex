\documentclass[12pt]{article}

\usepackage{setspace}
\usepackage{minted}
\usepackage{fancyvrb}

\usepackage[margin=0.75in]{geometry}
\pagestyle{empty}

\def \name       {Enrique Gavidia}
\def \coursenum  {CSC 415}
\def \coursename {Operating Systems Principles}
\def \instructor {Prof. Murphy}
\def \semester   {Spring 2012}
\def \assignment {Homework \# 1}
\def \duedate    {February 8, 2012}

\newcommand {\includesource}[2] {\inputminted[linenos, fontsize=\scriptsize, frame=lines]{#1}{#2}}
\newcommand {\append}[2] {\section*{Appendix #1} \textsl{\large #2}}
\newcommand {\filename}[1] {\flushleft \textbf{#1}}
\newcommand {\comment}[1] {\textcolor{red}{#1}}
\newcommand {\mytilde} {$\sim$}

\DefineVerbatimEnvironment {shelloutput} {Verbatim} {fontsize=\scriptsize, numbers=left, commandchars=\\\{\}}


\begin{document}

%---{ Cover Page }------------------------------------------
\begin{titlepage}
  \begin{center}

    {\LARGE \coursenum, \semester \\ \coursename}\\
    {\Large \instructor}\\

    \vfill
    \textbf{\Huge \assignment}\\
    \vfill
    
    {\Large \name}\\ 
    {\large \duedate}
    
  \end{center}
\end{titlepage}


%---{ Main Content }----------------------------------------



%---{ Appendices }------------------------------------------
\append{I} {Unix Source Code}
\includesource{c}{unix_copy.c}


\append{II} {Unix Output}
\subsection*{Small Files}

\filename{smallfile1}
\begin{shelloutput}
oblique@netlab4 \mytilde/hw1 $ ./unix_copy
Welcome to the TEE Copy Program by Enrique Gavidia!
Enter the name of the file to copy from:
smallfile1
Enter the name of the file to copy to:
smallcopy1

\comment{The buffer dumps in the debug output contain a maximum of 11 characters, as was designated in `buffer\_size' }
--------
BUFFER: abcdefghijk
--------
abcdefghijk
--------
BUFFER: lmnopqrstuv
--------
\comment{The normal printout of the buffer has no additional formatting, so as to act as a continuous stream to `stdout'} 
\comment{and replicate the contents of the file exactly on the user's screen. Here however, it is being interrupted by }
\comment{the debug output.}
lmnopqrstuv
--------
BUFFER: wxyz
123456
--------
wxyz
123456
--------
BUFFER: 7890

--------
7890
File copy successful, 38 bytes copied

\comment{The `-s' flag makes diff report when files are the same}
oblique@netlab4 \mytilde/hw1 $ diff -s smallfile1 smallcopy1
Files smallfile1 and smallcopy1 are identical
\end{shelloutput}


\filename{smallfile2}
\begin{shelloutput}
oblique@netlab4 \mytilde/hw1 $ ./unix_copy
Welcome to the TEE Copy Program by Enrique Gavidia!
Enter the name of the file to copy from:
smallfile2
Enter the name of the file to copy to:
smallcopy2

--------
BUFFER: This is a t
--------
This is a t
--------
BUFFER: est for the
--------
est for the
--------
BUFFER:  csc 415 fi
--------
 csc 415 fi
--------
BUFFER: le-copy ass
--------
le-copy ass
--------
BUFFER: ignment.

--------
ignment.
File copy successful, 53 bytes copied

oblique@netlab4 \mytilde/hw1 $ diff -s smallfile2 smallcopy2
Files smallfile2 and smallcopy2 are identical
\end{shelloutput}


\filename{smallfile3}
\begin{shelloutput}
oblique@netlab4 \mytilde/hw1 $ ./unix_copy
Welcome to the TEE Copy Program by Enrique Gavidia!
Enter the name of the file to copy from:
smallfile3
Enter the name of the file to copy to:
smallcopy3

--------
BUFFER: meow meow m
--------
meow meow m
--------
BUFFER: eow meow me
--------
eow meow me
--------
BUFFER: ow meow meo
--------
ow meow meo
--------
BUFFER: w meow

--------
w meow
File copy successful, 40 bytes copied

oblique@netlab4 \mytilde/hw1 $ diff -s smallfile3 smallcopy3
Files smallfile3 and smallcopy3 are identical
\end{shelloutput}



\subsection*{Large Files}

\filename{Makefile}
\begin{shelloutput}
oblique@netlab4 \mytilde/hw1 $ ./unix_copy
Welcome to the TEE Copy Program by Enrique Gavidia!
Enter the name of the file to copy from:
Makefile
Enter the name of the file to copy to:
Makefile_copy
\comment{The debug output is off for the larger files, so the buffer is dumped without any additional formatting,}
\comment{preserving the file's original formatting, and replicating it's contents exactly on the screen.}
CFLAGS=-Wall -Wextra -g

FILES = unix_copy

all: $(FILES)

clean:
	rm -f $(FILES)

with-clean: clean
	make
File copy successful, 107 bytes copied

oblique@netlab4 \mytilde/hw1 $ diff -s Makefile Makefile_copy
Files Makefile and Makefile_copy are identical
\end{shelloutput}


\filename{unix\_copy.c}
\begin{shelloutput}
oblique@netlab4 \mytilde/hw1 $ ./unix_copy
Welcome to the TEE Copy Program by Enrique Gavidia!
Enter the name of the file to copy from:
unix_copy.c
Enter the name of the file to copy to:
unix_copy2.c


oblique@netlab4 \mytilde/hw1 $ diff -s unix_copy.c unix_copy2.c
Files unix_copy.c and unix_copy2.c are identical
\end{shelloutput}

\subsection*{Strace}

\append{III} {Windows Source Code}
%\includesource{c}{win_copy.c}


\append{IV} {Windows Output}
\subsection*{Small Files}
\subsection*{Large Files}
\subsection*{StraceNT}

\end{document}
